%! Author = simon
%! Date = 2/9/25

%----------------------------------------------------------------------------------------
%	FONTS
%----------------------------------------------------------------------------------------
    \section{Font Examples}

    \subsection{Font Sizes}

    {\tiny \textbackslash tiny} {\scriptsize \textbackslash scriptsize} {\footnotesize \textbackslash footnotesize} {\small \textbackslash small}\\
    {\normalsize \textbackslash normalsize}\nonumsidenote[-1cm]{The default font size for the document is 12pt, represented by \textbackslash normalsize. The standard LaTeX font size commands modify this to be smaller or larger as needed.}\\
    {\large \textbackslash large} {\Large \textbackslash Large} {\LARGE \textbackslash LARGE} {\huge \textbackslash huge} {\Huge \textbackslash Huge}

    \subsection{Font Families}

    \textsf{IBM Plex Sans Text}\nonumsidenote{The sans family is the default, as is standard in the business world. Use the serif family to accentuate text, such as for quotations. The mono family is best used where it's important that all characters are the same width, such as for numbers in a table or for code.}

    \textrm{IBM Plex Serif Text}

    \texttt{IBM Plex Mono Text}

    \subsection{Font Weights}

    \textel{ExtraLight} \textl{Light} Normal \textsb{SemiBold} \textbf{Bold}

    \subsection{Condensed Fonts}

    Plex Sans Normal\nonumsidenote{Condensed fonts can be useful if horizontal space is at a premium. You might want to use the condensed font in a wide table.}

    {\plexsanscondensed Plex Sans Condensed}

